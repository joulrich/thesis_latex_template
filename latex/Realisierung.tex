\documentclass[thesis.tex]{subfiles}

\begin{document}

\chapter{Realisierung}\label{chap:realisierung}

\section{Vorbereitung}
\subsection{Technologien}\label{chap:technologien}
\subsection{Technische Grundlagen}
\section{Implementierung}
\subsection{Entwicklungsprozess}

Für die Umsetzung und Implementierung den Prototypen wurde ein agiler Entwicklungsprozess genutzt.
Dafür wurden kontinuierliche Iterationen über alle Phasen des Projekts verwendet.
Als ersteres wurde überlegt wie die bestehende Body-Cam an die Aufgabe der Überwachung bei Alleinarbeit angepasst werden kann.
Diese Anpassungen wurden als Anforderungen aufgeschrieben und in mehreren Iterationen weiter ausgeführt und verbessert.
Anschließend wurden konkrete Funktionen herausgearbeitet, die durch den neuen Lone-Worker-Modus abgedeckt werden sollen.
Es wurden verschiedene Statemachines(Fußnote oder auf deutsch lassen?) entwickelt die einen ersten Programmablauf beschreiben.
Dieser ging über Verbindungsaufbau, -abbau und -wiederherstellung, den Umgang mit Ein- und Ausgeheden Nachrichten sowie der Einbindung in die bereits bestehende Body-Cam Implementierung.
Danach wurden das Protokoll für die Kommunikation zwischen Gerät und Überwachungszentrum sowie die Primitiven zwischen Body-Cam und Lone-Worker-Modus herausgearbeitet.
Nach der Designphase begann die Implementierung.
Hier wurde sich als Erstes anhand der Anforderungen und gebeten Technologien der bestehenden Implementierung für die Technologien aus Kapitel \autoref{chap:technologien} entschieden.
Außerdem wurde im Rahmen der Arbeit die Implementierung des Lone-Worker-Modus auf die Grundfunktionalität des Verbindungsaufbaus, -abbaus und -wiederherstellung begrenzt.
Einzelheiten zur konkreten Umsetzung folgen im nächsten Kapitel.
\\

(Bin mir nicht sicher ob das mit rein muss aber es gehört ja zum Entwicklungsprozess)
Zur Entwicklung des Prototyps kam die Programmiersprache C++ sowie ein Linux Ubuntu 20.04 LTS zum Einsatz.
Auf diesem wurden dann alle nötigen Bibliotheken gebaut und mithilfe von CMake eine Projektstruktur entworfen.
CMake kümmert sich außerdem im Prozess des Compiling und Linking um die Verwaltung und Übersetzung der verschiedenen Quelldateien
sowie die Einbindung und Verknüpfung der Header und Bibliotheken.
Dabei wurde der Compiler G++ in der Version 9.4.0 genutzt.
Als Programmiereditor wurde Visual Studio Code mit der Erweiterung "clangd" verwendet.
Zur Versionsverwaltung wurde Git genutzt, mit einem eigenen Repository auf GitHub.

\subsection{Umsetzung}
\subsection{Herausforderungen bei der Implementierung}
\section{Ergebnisse}
\subsection{Testprotokoll}
\subsection{Erfüllung der Anforderungen}

\subfilebib % Makes bibliography available when compiling as subfile
\end{document}
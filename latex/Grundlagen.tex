\documentclass[thesis.tex]{subfiles}

\begin{document}

\chapter{Grundlagen}\label{chap:grundlagen}

\section{Aktuelle Alleinarbeitslösungen}

Im Arbeitsalltag treten immer wieder Situation auf die zur Gefährdung von Personen führt.
Diese Situationen ergeben sich aus den verschiedenen Arbeitsverfahren, Tätigkeiten, Werkstoffen sowie Umgebungen die in allen Arbeitsbereichen unterschiedlich zur Geltung kommen.
Arbeiten die eine deutliche Gefährdung in diesen Faktoren aufweisen lässt sich als "Gefährliche Arbeit" klassifizieren.
Beispiele dafür sind das Arbeiten in engen Räumen wie Silos oder Containern, Feuerarbeiten in brand- oder explosionsgefährdeten Bereichen oder in geschlossenen Räumen, Gleisarbeit während des Bahnbetriebes, Feuerwehr, Tunnelbau, Umgang mit besonders gefährlichen Stoffen aber auch eine Dienstleistung an Personen zu verüben die sich gegen diese tätlich wehren.
Letzteres kommt besonders bei der Polizei oder in sozialen Bereichen vor.
100-001 2.7.1

In den meisten Fällen sollten gefährliche Arbeiten von mehreren Personen gemeinschaftlich ausgeführt werden.
Zur Vermeidung von Gefahren muss die Aufsichtspflicht durch eine zuverlässige, mit der Arbeit vertraute Person durchgeführt werden, die die gegenseitige Verständigung koordiniert, um die arbeitssichere Durchführung zu gewährleisten \cite[§8.1]{V1DGUV2013}.
In Ausnahmen kann es aus betrieblichen Umständen notwendig sein, eine einzelne Person mit dieser Art von Arbeit zu betrauen. Dies ist grundsätzlich zulässig und kommt im Arbeitsalltag vielfach vor.
Alleinarbeit liegt laut 100.001 Vorschrift 2.7.2 vor, wenn eine Person allein, außerhalb von Ruf- und Sichtweite zu anderen Personen, Arbeiten ausführt.
Wird eine gefährliche Arbeit als Alleinarbeit vollzogen, muss ein Unternehmer über die allgemeinen Schutzmaßnahmen hinweg für ausreichend technische oder organisatorische Personenschutzmaßnahmen sorgen.
Zu diesen Maßnahmen zählen beispielsweise der Einsatz von Personen-Notsignal-Anlagen (PNA), ununterbrochene Kameraüberwachung, Kontrollgänge einer zweiten Person sowie kontinuierliche Meldungen durch Telefon- und Funksystemen. 100-001
Dies soll Sicher stellen das in einem Notfall der Alleinarbeiter so schnell wie möglich einen Notruf absetzten kann.
Die Notrufmöglichkeiten sind ein wichtiger Teil der Rettungskette.
Die Notwendigkeit verschiedenen Möglichkeiten zur Gefahrenminimierung werden durch die Gefährdungsstufen, Erstversorgungszeit und die Notfallwahrscheinlichkeit festgelegt. 112-139

DGUV Information 212-139 ABB 1 !! Passt hier denke sehr gut hin.

Wie in der Abbildung ersichtlich wird, kann bei Alleinarbeit die Gefährdung in 3 verschiedene Stufen eingeteilt werden.
Geringe Gefährdung heißt dabei das die allein arbeitende Person nach einem schädlichen Ereignis, sich nur geringe Verletzungen zuziehen kann bzw. nur geringe akute Beeinträchtigung der Gesundheit verursacht werden.
Die Person bleibt also nach einem Unfall noch voll handlungsfähig.
Wie der Grafik zu entnehmen ist, ist dies der einfachste Fall.
Hier reicht eine Festnetzverbindung aus um z.B. den Notruf wählen zu können.
Die zweite Stufe ist die der erhöhten Gefährdung.
Bei dieser kann der Alleinarbeiter erhebliche Verletzungen bzw. starke akute Beeinträchtigung der Gesundheit erleiden.
In diesem Fall bleibt die Person nach einem Notfall nur eingeschränkt handlungsfähig.
Bei einer hohen Wahrscheinlichkeit eines schädlichen Ereignisses muss diese Art von Tätigkeit sogar als kritische Gefährdung betrachtet werden.
In Bezug auf den Einsatz einer geeigneten Meldeeinrichtung dürfen jetzt keine stationären bzw. leitungsgebundenen Einrichtungen mehr benutzt werden.
Je nach Arbeit können hier aber Mobiltelefone, Sprechfunkgeräte, zeitgesteuerte Kontrollanrufe, durchgehende Kameraüberwachung oder PNAs zum Einsatz kommen.
Die letzte Stufe ist die kritische Gefährdung.
In der höchsten der drei Stufen können sich allein arbeitende Personen besonders schwere Verletzungen bzw. Beeinträchtigung der Gesundheit zu ziehen.
Sie sind nach einer schädlichen Situation als vollständig handlungsunfähig.
Bei dieser Stufe ist eingehend zu prüfen Alleinarbeit überhaupt zulässig ist.
Anderen Falls kann das Unternehmen an diesem Ort unter den gegebenen Umständen keine Alleinarbeit vollführen.
Hier kommen als Meldeeinrichtung nur noch die ständige Kameraüberwachung in Verbindung mit einem PNA infrage.

LKW im Werksverkehr, Reinigungsfachkräfte und Taxifahrt während der Tagesschicht fallen dabei unter die geringe Stufe.
Gefahrguttransport, Jugendhilfe und Dacharbeiten lassen sich in die zweite Kategorie einordnen.
Als kritische Gefährdung zählen Werttransporte, Aufzugmontage und -instandhaltung und Taxifahrten bei Nacht.
Grundsätzlich ist diese Einteilung nur eine grobe Einordnung in die unterschiedlichen Stufen.
Bei jeder Arbeit muss eine individuelle Abschätzung der Gefährdung gemacht werden, da wie oben erwähnt Art der Tätigkeit, Umgebung und andere Faktoren eine große Rolle auf die Sicherheit haben.
Gleiches gilt für den Einsatz von den verschiedenen Meldeeinrichtungen in den Stufen.
Hier muss ebenfalls abgewägt werden welche Einrichtung für den Einsatz ausreichend geeignet ist. 212-139

Um die Gefährdung einer Arbeit genau festlegen zu können, gibt es gemäß DGUV Regel 112-139, die Bewertung der Arbeiten in verschiedene Schutzniveaus.
Dabei spielen sowohl die eben beschriebenen Gefährdungsstufen, die Zeit bis zu einer Erstversorgung und die Wahrscheinlichkeit eines Notfalls eine Rolle.
Mithilfe dieser Faktoren lässt sich ein Schutzniveau errechnen, welches für die Einteilung der Arbeiten genutzt wird.
Nach dieser Einteilung werden dann Maßnahmen festgelegt, um die Sicherheit zu erhöhen, die Notfallkette zu gewährleisten oder um ein Verbot von Alleinarbeit zu bestimmen.


\subfilebib % Makes bibliography available when compiling as subfile
\end{document}
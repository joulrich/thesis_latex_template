\documentclass[thesis.tex]{subfiles}

\begin{document}

\chapter{Grundlagen}\label{chap:grundlagen}

\section{Aktuelle Alleinarbeitslösungen}

Im Arbeitsalltag treten immer wieder Situation auf die zur Gefährdung von Personen führt.
Diese Situationen ergeben sich aus den verschiedenen Arbeitsverfahren, Tätigkeiten, Werkstoffen sowie Umgebungen die in allen Arbeitsbereichen unterschiedlich zur Geltung kommen.
Arbeit die eine deutliche Gefährdung in diesen Faktoren aufweist lässt sich als "Gefährliche Arbeit" klassifizieren.
Beispiele dafür sind das Arbeiten in engen Räumen wie Silos oder Containern, Feuerarbeiten in brand- oder explosionsgefährdeten Bereichen oder in geschlossenen Räumen, Gleisarbeit während des Bahnbetriebes, Feuerwehr, Tunnelbau, Umgang mit besonders gefährlichen Stoffen aber auch eine Dienstleistung an Personen zu verüben die sich gegen diese tätlich wehren.
Letzteres kommt besonders bei der Polizei oder in sozialen Bereichen vor \cite[vgl. S.41 2.7.1]{Regel_100-001}.

In den meisten Fällen sollten gefährliche Arbeiten von mehreren Personen gemeinschaftlich ausgeführt werden.
Zur Vermeidung von Gefahren muss die Aufsichtspflicht durch eine zuverlässige, mit der Arbeit vertraute Person durchgeführt werden, die die gegenseitige Verständigung koordiniert, um die arbeitssichere Durchführung zu gewährleisten \cite[vgl. §8.1]{Vorschrift1_DGUV}.
In Ausnahmen kann es aus betrieblichen Umständen notwendig sein, eine einzelne Person mit dieser Art von Arbeit zu betrauen. Dies ist grundsätzlich zulässig und kommt im Arbeitsalltag vielfach vor.
Alleinarbeit liegt laut 100.001 Vorschrift 2.7.2 vor, wenn eine Person allein, außerhalb von Ruf- und Sichtweite zu anderen Personen, Arbeiten ausführt.
Wird eine gefährliche Arbeit als Alleinarbeit vollzogen, muss ein Unternehmer über die allgemeinen Schutzmaßnahmen hinweg für ausreichend technische oder organisatorische Personenschutzmaßnahmen sorgen.
Zu diesen Maßnahmen zählen beispielsweise der Einsatz von Personen-Notsignal-Anlagen (PNA), ununterbrochene Kameraüberwachung, Kontrollgänge einer zweiten Person sowie kontinuierliche Meldungen durch Telefon- und Funksystemen \cite[vgl. S.43 2.7.2]{Regel_100-001}.
Dies soll sicherstellen das in einem Notfall der Alleinarbeiter so schnell wie möglich einen Notruf absetzten kann.
Die Notrufmöglichkeiten sind ein wichtiger Teil der Rettungskette.
Die Notwendigkeit verschiedenen Möglichkeiten zur Gefahrenminimierung werden durch die Gefährdungsstufen, Erstversorgungszeit und die Notfallwahrscheinlichkeit festgelegt \cite[vgl. S.13-18]{Regel_112-139}.

DGUV Information 212-139 ABB 1 !! Passt hier denke sehr gut hin.

Wie in der Abbildung ersichtlich wird, kann bei Alleinarbeit die Gefährdung in 3 verschiedene Stufen eingeteilt werden.
Geringe Gefährdung heißt dabei das die allein arbeitende Person nach einem schädlichen Ereignis, sich nur geringe Verletzungen zuziehen kann bzw. nur geringe akute Beeinträchtigung der Gesundheit verursacht werden.
Die Person bleibt also nach einem Unfall noch voll handlungsfähig.
Wie der Grafik zu entnehmen ist, ist dies der einfachste Fall.
Hier reicht eine Festnetzverbindung aus um z.B. den Notruf wählen zu können.
Die zweite Stufe ist die der erhöhten Gefährdung.
Bei dieser kann der Alleinarbeiter erhebliche Verletzungen bzw. starke akute Beeinträchtigung der Gesundheit erleiden.
In diesem Fall bleibt die Person nach einem Notfall nur eingeschränkt handlungsfähig.
Bei einer hohen Wahrscheinlichkeit eines schädlichen Ereignisses muss diese Art von Tätigkeit sogar als kritische Gefährdung betrachtet werden.
In Bezug auf den Einsatz einer geeigneten Meldeeinrichtung dürfen jetzt keine stationären bzw. leitungsgebundenen Einrichtungen mehr benutzt werden.
Je nach Arbeit können hier aber Mobiltelefone, Sprechfunkgeräte, zeitgesteuerte Kontrollanrufe, durchgehende Kameraüberwachung oder PNAs zum Einsatz kommen.
Die letzte Stufe ist die kritische Gefährdung.
In der höchsten der drei Stufen können sich allein arbeitende Personen besonders schwere Verletzungen bzw. Beeinträchtigung der Gesundheit zu ziehen.
Sie sind nach einer schädlichen Situation als vollständig handlungsunfähig.
Bei dieser Stufe ist eingehend zu prüfen ob Alleinarbeit überhaupt zulässig ist.
Anderen Falls kann das Unternehmen an diesem Ort unter den gegebenen Umständen keine Alleinarbeit vollführen.
Hier kommen als Meldeeinrichtung nur noch die ständige Kameraüberwachung in Verbindung mit einem PNA infrage.

LKW im Werksverkehr, Reinigungsfachkräfte und Taxifahrt während der Tagesschicht fallen dabei unter die geringe Stufe.
Gefahrguttransport, Jugendhilfe und Dacharbeiten lassen sich in die zweite Kategorie einordnen.
Als kritische Gefährdung zählen Werttransporte, Aufzugmontage und -instandhaltung und Taxifahrten bei Nacht.
Grundsätzlich ist diese Einteilung nur eine grobe Einordnung in die unterschiedlichen Stufen.
Bei jeder Arbeit muss eine individuelle Abschätzung der Gefährdung gemacht werden, da wie oben erwähnt Art der Tätigkeit, Umgebung und andere Faktoren eine große Rolle auf die Sicherheit haben.
Gleiches gilt für den Einsatz von den verschiedenen Meldeeinrichtungen in den Stufen.
Hier muss ebenfalls abgewägt werden welche Einrichtung für den Einsatz ausreichend geeignet ist \cite[vgl. S.7-9]{Information_212-139}.

Um die Gefährdung einer Arbeit genau festlegen zu können, gibt es gemäß DGUV Regel 112-139 \cite[]{Regel_112-139}, die Bewertung der Arbeiten in verschiedene Schutzniveaus.
Dabei spielen sowohl die eben beschriebenen Gefährdungsstufen, die Zeit bis zu einer Erstversorgung und die Wahrscheinlichkeit eines Notfalls eine Rolle.
Mithilfe dieser Faktoren lässt sich ein Schutzniveau errechnen, welches für die Einteilung der Arbeiten genutzt wird.
Nach dieser Einteilung werden dann Maßnahmen festgelegt, um die Sicherheit zu erhöhen, die Notfallkette zu gewährleisten oder um ein Verbot von Alleinarbeit zu bestimmen.

Bei Betrachtung der einzelnen Meldeeinrichtungen fällt das PNA auf, da es als einziges ohne zusätzliche Maßnahmen in allen drei Stufen eingesetzt werden kann.
Außerdem fällt das Mobiltelefone bzw. Smartphone auf.
Da dieses Gerät sehr weit verbreitet ist und es viele Menschen im Alltag verwenden.
Mit ihm kann immerhin die geringe und Teile der erhöhten Gefährdung gut abgedeckt werden.
Sie eignen sich nicht für die kritische Stufe, da bestimmte Hardware-Anforderungen, die in der Industrie benötigt werden, nicht erfüllt sind.
Darunter zählen der ausreichende Schutz vor Wasser, Staub, Chemikalien und rauem Umgang sowie der Gebrauch in sehr hohen oder niedrigen Temperaturen.
Die Alarmauslösung muss zu jeder Zeit gewährleistet werden.
Dazu müssen Geräte in dieser Gefährdungsstufe einen hardwareseitigen, roten Alarmknopf besitzen.
Diese Eigenschaften erfüllen normale Smartphones aus dem Handel nicht.
Dafür können sie sehr einfach in geringeren Stufen eingesetzt werden.
In diesem Fall kann sich der Einsatz einer Notfall-App anbieten.
Diese überträgt ein Signal an eine vorher festgelegte Stelle, zum Anzeigen eines Notfalls und um bei Bedarf die Rettungskräfte zu alarmieren.
Bei guter Programmierung kann die App den Einsatzkomfort des Smartphones in Notfallsituation deutlich erhöhen und die Belastung auf den Arbeiter reduzieren \cite[vgl. S.2-5]{FAQ-PNAuAPP}.

Das PNA kann in allen Bereichen eingesetzt werden und erfüllt auch die Industrieanforderungen.
Es benutzt öffentliche Mobilfunknetze um eine dauerhafte Verbindung zu einer Empfangseinrichtung zu gewährleisten.
Durch die Hardware-Notruftaste, eine Lokalisierung über GPS, Überwachung der Verbindung sowie des Geräte-Status und vorher definierten Alarmparametern, stellt sie die umfangreichste Lösung für Alleinarbeit dar.
Unter Alarmparameter fallen beispielsweise professionelle Alarmbearbeitung durch eine Notrufzentrale, wiederholte Alarmübertragung und Voralarme \cite[vgl. S.2-5]{FAQ-PNAuAPP}.
Alle Bauanforderungen und Funktionen die eine PNA erfüllen muss sind in der DGUV Regel 112-139 \cite[]{Regel_112-139} beschrieben.
Diese beruht auf der Norm "DIN VDE V 0825-11: Geräte- und Prüfanforderungen für Personen-Notsignal-Anlagen unter Nutzung öffentlicher Telekommunikationsnetze".
Die als Gegenstand dieser Arbeit vorgestellte Body-Cam ist in die Kategorie PNA einzuordnen.
Genauere Details der einzelnen Funktionen werden in Kapitel \textcolor{red}{xxx} beschrieben.
\\

\section{IST-Zustand Body-Cam}

Die Body-Cam ist das Hauptprodukt von NetCo und wurde 2016 entwickelt. Mittlerweile kommt sie europaweit erfolgreich zum Einsatz.
In erster Linie wurde sie zur Prävention und Deeskalation bei Konflikten entwickelt und kommt bei Polizei, Ordnungsämtern, Verkehrsbetrieben und Sicherheitsdiensten zum Einsatz.
Hier vor allem zur Dokumentation von Einsätzen und zur rechtskonformen Beweissicherung und deren Gerichtsverwertbarkeit.
\\

Die Body-Cam ist mit ihren Maßen von 102x62x25 mm kompakt gehalten und lässt sich daher gut während unterschiedlicher Arbeiten am Körper tragen ohne zu behindern.
Außerdem hat sie trotz der relativ geringen Größe ein 2,8 Zoll (ca. 7 cm) Farbdisplay mit integrierter Touchfunktion und einer Weitwinkelkamera. Dadurch lassen sich sehr einfach alle Statusinformationen einfach ablesen und es können sowohl Textnachrichten als auch der aktuelle Videostream angezeigt werden.
Der Touchscreen lässt sich auch problemlos mit Handschuhen bedienen was gerade im Einsatz sehr wichtig ist.
Sie kann in einem Temperaturbereich von -20 Grad bis + 50 Grad Celsius betrieben werden.
Zusätzlich besitzt sie die Schutzklasse IP 65 und ist damit dicht gegen Staub und geschützt gegen Strahlwasser (Elektronische Gerätetechnik/Book Subtitle:Grundlagen für das Entwickeln elektronischer Baugruppen und Geräte/Authors:Jens Lienig, Hans Brümmer/https://link.springer.com/book/10.1007/978-3-642-40962-2) (Datenblatt Body-Cam).
Die Kamera zur Videoaufzeichnung hat ein 160 Grad Weitwinkel-Objektiv und verfügt über eine hohe Lichtstärke.
Damit kann sie die Situation des Trägers in einem großen Bereich erfassen und auch Aufnahmen bei schlechter Beleuchtung oder bei Dämmerung gut darstellen.
Der Akku versorgt die Body-Cam sowie eine integrierte helle LED-Lampe mit Strom.
Die Kapazität der Batterie beträgt dabei 5.000 mAh.
Dieser große Akku ist notwendig um eine Versorgung über eine ganze Schicht, also über 8 Stunden, zu gewährleisten.
Die Aufladung der Batterie erfolgt dabei über die zu der Body-Cam entwickelten Docking-Station.
Diese dient gleichzeitig zum Aufladen ein oder mehrerer Kameras sowie zur Übertragung aller auf dem internen Speicher befindlichen Aufnahmen an ein zentrales System.
Die Body-Cam ist ebenfalls mit WLAN und Bluetooth ausgestattet.
WLAN kann in bestimmten Umgebungen die Mobilfunkverbindung ersetzten.
Beim Einsatz in Zügen der Deutschen Bahn lässt sich beispielsweise das im Zug integrierte WLAN nutzen, um bei der Zugfahrt unabhängig vom ständig wechselten Mobilfunknetz zu sein.
In Verbindung mit Geräten anderer Hersteller, wie z.B. einem Bluetooth-fähigem Waffenholster, kann die Body-Cam über Bluetooth mit diesem gekoppelt werden.
Hierdurch kann automatisch eine Aufnahme gestartet werden, wenn die Waffe aus dem Holster gezogen wird.
\\

Die NetCo Body -Cam wird grundsätzlich in zwei Versionen angeboten.
In der Record-Version wird die Aufnahme auf dem Gerät gespeichert und nach dem Einsatz automatisch über eine Docking-Station in das zentrale System überspielt.
Mit der 128 GB großen microSD-Karte lassen sich damit bis zu 17 Stunden Aufnahmezeit erreichen.
Die Aufnahmen werden dabei mit AES 256 Bit verschlüsselt.
Außerdem ist auch eine Voraufzeichnungsdauer von bis zu 2 Minuten möglich.
Die Voraufzeichnung findet dabei in einer Schleife statt und wird erst nach Bestätigen der Aufnahmetaste der Aufnahme vorangestellt.
Dies dient zur Sicherstellung des Datenschutzes und hat zugleich den Vorteil das der Träger der Body-Cam in Gefahrensituationen erst handeln und dann die Aufnahmetaste drücken kann ohne das dabei Informationen verloren gehen.
\\

In der Connect-Version erfolgt die unmittelbare Übertragung über ein gesichertes VPN an das zentrale System.
Dabei hat die Kamera eine 4G LTE Verbindung und kann die Aufnahmen nach Beendigung direkt an die zuständige Dienststelle übermitteln.
Dies ist beispielsweise in Situationen wichtig in denen zu Befürchten ist, dass die Herausgabe der Kamera mit Gewalt erzwungen wird.
Hier kann im Zweifel die Kamera herausgegeben werden, um eine weitere Eskalation zu vermeiden, ohne das dabei Aufnahmematerial verloren geht.
Mit dem übertragenden Material kann dann umgehend eine Fahndung und Unterstützung eingeleitet werden.
Beide Systemvarianten decken die Zielstellungen Prävention, Deeskalation und Dokumentation zuverlässig ab.

\section{Erweiterung der Body-Cam mit dem Alleinarbeits-Modus}

Die Body-Cam wurde für die Prävention und Deeskalation in kritischen Situation entwickelt.
Durch den Einsatz bei Polizei, Ordnungsämtern, Verkehrsbetrieben und Sicherheitsdiensten und der steigenden Nachfrage der letzten Jahre wurde die Body-Cam in diesen Punkten noch weiter verbessert.
Sie bringt damit sehr viele Eigenschaften bereits mit dem sie als ein PNA für Alleinarbeit nach DGUV Regel 112-139 \cite[]{Regel_112-139} qualifiziert.
So besitzt die Body-Cam beispielsweise eine ausreichende Schutzklasse, ist mit Mobilfunk, WLAN sowie Bluetooth ausgestattet, kann bei sehr hohen und niedrigen Temperaturen betrieben werden und lässt sich einfach und zuverlässig in schwierigen Situation bedienen.
Außerdem bringt Sie weitere Eigenschaften mit die für ein PNA nicht direkt gefordert sind aber die für die Alleinarbeit nützlich sind.
Hierbei wären z.B. die Kamera zu nennen mit der eine dauerhafte Überwachung durch einen Videostream möglich ist sowie das vergleichsweise große Display welches einen guten Überblick über alle Statuswerte verschafft.
\\

Es gibt jedoch auch Hardwarekomponenten und Software die nachgerüstet werden müssen, um einen vollständigen Einsatz als PNA zu gewährleisten.
Einer der wichtigsten Eigenschaften ist das Auslösen eines Alarms.
Dieser soll in Gefahrensituationen einfach durch einen hardwareseitigen roten Knopf zuverlässig ausgelöst werden können.
Der rote Knopf ist bereits in der Body-Cam als Aufnahmeknopf verbaut und muss für die Lone-Worker Version nur mit der entsprechenden Alarmfunktion verknüpft werden.
Als Zweites muss noch ein Lagesensor nachgerüstet werden.
Dieser Sensor soll die Lage des LW kontinuierlich überwachen und feststellen, wenn sich diese plötzlich ändert.
Das wird genutzt, um festzustellen, ob der LW beispielsweise ohnmächtig in sich zusammengebrochen ist und damit handlungsunfähig ist.
Bei Eintreten dieses Ereignisses wird erst ein Voralarm ausgelöst, diesen kann der Träger bei Falschmeldung abstellen.
Anschließend wird ein richtiger Alarm an das Überwachungszentrum gesendet.
Dieser informiert, dass der LW möglicherweise verunglückt und handlungsunfähig ist.
\\

Als Software um alle Funktionen bereitzustellen und zu überwachen muss der Lone-Worker-Modus eingeführt werden.
Dieser ist Hauptbestandteil dieser Arbeit und wird in folgenden Kapiteln ausführlich beschrieben, designt und als Prototyp mit seiner Grundfunktionalität  implementiert.
Grundsätzlich ist dieser dafür zuständig eine Verbindung zu einem Überwachungszentrum herzustellen und dauerhaft zu halten oder bei Bedarf umgehend wieder herzustellen.
Er übernimmt die Kommunikation zwischen Überwachungszentrum und Body-Cam mithilfe eines eigenen Protokolls.
Der Modus verwaltet alle Alarme, Voralarme und Alarmwiederholungen sowie die Statuswerte des Geräts.
Über verschiedene Schnittstellen, die der Modus zur Body-Cam hat, kann er dem Überwachungszentrum Geräteinformationen, Standorte, Video- und Audiostreams sowie einen Sprachkanal ununterbrochen zur Verfügung stellen.
Audio- und Videodaten sowie Standortinformationen werden dabei bereits durch Kamera, Mikrofon und GPS erfasst und müssen zur in die neue Funktion eingebunden werden.
Der Lone-Worker-Modus erweitert also die bestehen Funktionen der Body-Cam zur Prävention, Deeskalation und Protokollierung um eine Möglichkeit sie als PNA für Alleinarbeit zuverlässig einzusetzen.
\\

\section{Beschreibung des Gesamtsystems}

\subsection{Komponenten}
Das Gesamtsystem setzt sich hauptsächlich zusammen aus der Body-Cam (BC) als Clientgerät, dem „Lone Worker“(LW)
als Träger der Body-Cam sowie einem Überwachungszentrum als Server (MC).
%%
Die BC baut nach einschalten eine Verbindung mit einer Gegenstelle auf und versucht diese zu halten.
%%
Hierbei ist der LW-Modus der BC noch nicht aktiv und es findet noch keine ununterbrochene Überwachung statt.
%%
\\

Das Grundszenario kann wie folgt beschrieben werden. Der LW kann nun aktiv den LW-Modus auf seiner BC aktivieren.
%%
Damit stellt die BC eine Anfrage an ein MC für eine LW-Verbindung mit ununterbrochener Überwachung.
%%
Wenn diese durch das MC bestätigt ist, gehen beide Parteien in den LW-Modus über.
%%
Dies hat zur Folge das nach vorher festgelegten Parametern eine strenge Überwachung der Verbindung stattfindet.
%%
Bei Verbindungsverlust wird sowohl der LW als auch dem MC mitgeteilt das die ununterbrochene Überwachung nicht mehr gewährleistet werden kann.
%%
Außerdem kann das MC jederzeit Informationen über die BC sowie dessen Träger erfragen.
%%
So können Informationen über Gerätestatus, letzte Standorte oder Audio- und Videostreams durch das MC angefordert werden.
%%
Neben diesem Grundszenario gibt es noch viele weitere Möglichkeiten die später in den Use-Cases genauer beleuchtet werden.
%%
\\

Die Body-Cam ist also ein Embedded-Gerät, welches zur Dokumentation und Beweissicherung von Konflikt- und Gefahrensituationen dient.
%%
Sie besitzt eine Kamera zur Videoaufzeichnung sowie ein großes Display zur Bedienung und um die deeskalierende Wirkung der BC zu verstärken.
%%
Vor allem durch die Platzierung am Oberkörper des Trägers kann eine gute Sicht aus Perspektive einer Person gewährleistet werden.
%%
Mit dem neuen Feature des „Lone Worker“-Modus soll die BC in ihrer Funktion erweitert und verstärkt werden.
%%
In Gefahrensituationen soll durch das neue Feature die Sicherheit gesteigert werden, gerade wenn der Träger allein ist.
%%
\\

Der „Lone Worker“ als Träger der Body-Cam soll durch den Modus seine Aufgaben besser und mit einer größeren Sicherheit erledigen können.
%%
Darunter zählen zum Beispiel die Überwachung und/oder Sicherung eines Bereiches, Kontrollgänge durch ein Gebäude oder
als Sicherheitspersonal in Einkaufszentren, Bahn, Schule und vielem mehr.
%%
Der Träger soll die Sicherheit haben durch die BC mit dem LW-Modus unter einer ununterbrochenen Überwachung zustehen,
die im Notfall schnell auf Gefahrensituationen reagieren kann.
%%
\\

Das Überwachungszentrum dient als Leitstelle wohin sich mehrere BC verbinden können.
%%
Je nach zur Verfügung stehenden Kapazitäten können dann strengere Überwachungen mit eingeschaltetem LW-Modus durchgeführt werden.
%%
Den Aufbau und Abbau von Verbindungen sowie das Anfordern und Darstellen von Informationen über die einzelnen BCs wird
durch einen Automatismus realisiert.
%%
Die Bestätigung des LW-Modus mit Übergang in die ununterbrochene Überwachung muss jedoch aktiv durch einen Mitarbeiter des MC stattfinden. (Ergänzen warum das gemacht werden muss oder ändern)
\\

\subsection{Use-Cases}
\section{Anforderungen des Gesamtsystems}
Aus den Punkten des Lastenheftes (ändern da kein Lastenheft als direkte Quelle existiert) lassen sich folgende Anforderungen für das Gesamtsystem herausarbeiten.
\\

Die BC sollte einen Schutz gegen versehentliche Aktivierung des LW-Modus besitzen.
%%
Dies lässt sich zum Beispiel wie folgt realisieren: Die Aktivierung eines Voralarms, der ohne Konsequenzen abgebrochen werden kann.
%%
Ein spezielles Design des Aktivierungstasters. Hier kann beispielsweise ein versenkter Taster benutzt werden für den mehr Kraft zur
Aktivierung benötigt wird.
%%
Die Dauer des Tastendrucks zur Aktivierung des Modus kann erhöht werden, sollte jedoch im Bereich von wenigen Sekunden sein.
%%
\\

Es sollte die Möglichkeit für eine „Two-Way-Audio“-Verbindung zwischen Client und Server bestehen.
%%
Dies schafft für den LW die Gelegenheit Kommentare der überwachenden Person zu hören und auf diese zu Antworten.
%%
Außerdem kann der LW nach mündlicher Aufforderung eine Aktion bestätigen oder Informationen über den Einsatzfortschritt geben.
%%
\\

Nach einer erfolgreichen Authentifizierung soll das MC Fernzugriff zum LW bekommen.
%%
Dies soll folgende Optionen ermöglichen: Abruf von Audio- und Videostreams, Abruf von auf dem Gerät gespeicherten Mediendateien
(Datengröße kann für LTE ggf. zu groß sein), Abruf von Geräteparametern und das setzten bestimmter Parameter, Lesen des Gerätestatus
sowie die Möglichkeit eine neue Firmware auf einzelne Geräte zu übertragen.
%%
Einige dieser Optionen wie das setzten von Geräteparametern oder ein Update der Firmware sollten nur bei ausgeschaltetem LW-Modus
und nach Bestätigung des LW möglich sein.
%%
\\

Der LW soll zu jeder Zeit die Möglichkeit haben Informationen über die Stärke des Netzwerksignals und den Batteriestatus am Gerät zu sehen.
%%
Hierfür soll der aktuelle Signalpegel und Batteriezustand auf dem Display zu sehen sein sowie ein haptisches, akustisches oder visuelles
Feedback bekommen, wenn keine Verbindung verfügbar ist oder der Batteriestand in einen kritischen Bereich fällt.
%%
Dem MC sollen diese Informationen sowie die Benachrichtigung bei kritischen Batteriezustand ebenfalls zur Verfügung stehen.
%%
\\

Dem Träger der BC soll die Gelegenheit haben einen „Timer“ einzustellen.
%%
Dieser dient dazu die Alarmfunktion temporär abzustellen.
%%
Diese Option ist vor allem sinnvoll, wenn es zu einer erwartenden Einschränkung des Kommunikationsnetzes kommt.
%%
Dies tritt beispielweise auf, wenn sich der LW in einen Fahrstuhl begibt.
%%
In dem Fall kann der Träger ein aussetzten der Verbindung in der BC vermerken.
%%
In dieser Zeit wird bei Verbindungsverlust kein Alarm auf beiden Seiten der Kommunikation ausgelöst.
%%
Das MC wird über das Einschalten dieser Funktion informiert.
%%
Wenn nach Ablauf der Zeit keine Verbindung zwischen BC und MC hergestellt werden kann, gehen beide Parteien in den Alarmzustand über.
%%
Der LW hat die Möglichkeit den Timer vor Ablauf der Zeit zu deaktivieren und damit wieder in die normale Funktionsweise des LW-Modus überzugehen.
%%
\\

Die BC soll dem MC auf Anfrage die aktuelle Position und/oder einer Liste der letzten 50 bekannten Positionen übermitteln.
%%
Dies wird vor allem dafür genutzt das Konfigurieren und Überwachen von „Geo-Fences“ zu ermöglichen.
%%
Damit können Ein- und Austreten aus bestimmten Bereichen überwacht werden und dem MC automatische Benachrichtigungen
über diese Vorgänge gemacht werden.
%%
Außerdem können auf der BC bestimmte Reaktionen konfiugriert werden die beim Betreten oder Verlassen von „Geo-Fence“-Gebieten aktiv werden.
%%
Zum Beispiel kann der LW-Modus aktiviert oder deaktiviert werden, automatisch ein Videostream an das MC übertragen werden
oder der LW wird explizit durch ein Feedback vor dem Betreten eines besonderen Gefahrengebiets gewarnt.
%%
\\

Das MC soll die Möglichkeit haben Statistiken über die einzelnen Geräte anfertigen können.
%%
Dafür müssen alle Aktivierungsmeldungen, Alarme, einschließlich Fehlalarme erfasst werden.
%%
Ebenfalls soll die Verfügbarkeit des Mobilfunknetztes aufgezeichnet werden.
%%
Diese Informationen müssen von der BC bereitgestellt werden und sind vom MC zu erfassen.
%%
Außerdem soll die BC alle Events dokumentieren und auf ihrer SD-Karte speichern.
%%
Darunter zählen die oben beschriebenen Auskünfte sowie alle Gerätestatusinformationen wie beispielsweise Batteriestatus
oder Speicherplatz, GPS-Standorte, Änderungen in der Konfiguration, Aktivierung und Deaktivierung des LW-Modus
sowie des temporären „Timer“-Modus, Anfragen und Antworten von und an das MC und Softwareinformationen.
%%
\\

Die BC soll dem LW ein haptisches, akustisches und visuelles Feedback geben können.
%%
Dies kann in Form eines Voralarms sein, wie oben beim Aktivieren des LW-Modus beschrieben oder beim Ausführen eines Alarms
und die Aufmerksamkeit des LW zu erhalten.
%%
Außerdem kann dadurch das MC dem LW durch ein haptisches Feedback einen „Stillen“-Alarm senden.
%%
Weiterhin soll das Display der BC für visuelles Feedback sowie das Anzeigen von Textnachrichten vom MC an den LW genutzt werden.
%%
Außerdem kann das MC über diese Schnittstelle Fragen an den LW stellen.
%%
Diese kann der Träger durch auf dem Display eingeblendete Optionen beantworten.
%%
Diese werden vom MC mit der Frage mitgeschickt und sind somit vorher festgelegt.
%%
Da das Display über eine Touch-Screen Funktion verfügt, kann er LW die Antwortmöglichkeiten mit Berührung des Displays auswählen.
%%
\\

Die BC soll die Möglichkeit bieten die Konfigurationsmöglichkeiten einzuschränken, um ungewollte oder unbefugte Änderungen zu verhindern.
%%
Dies dient in erster Linie der Sicherheit des Trägers und soll einen einfachen und schnellen Einsatz gewährleisten.
%%
So können die Geräte vorher durch ein Unternehmen oder eine Abteilung konfiguriert und anschließend an die einzelnen LW ausgeteilt werden.
\\


\subfilebib % Makes bibliography available when compiling as subfile
\end{document}
\documentclass[thesis.tex]{subfiles}
\begin{document}

\chapter{Einleitung}
\label{chap:Einleitung}

% An image right here at the top can look really cool!

\section{Einleitung}
Das Unternehmen NetCo Professional Services GmbH mit Sitz in Blankenburg und Magdeburg entwickelt und produziert individuelle Kamera- und Softwarelösungen.
Ein Hauptprodukt, die NetCo Body-Cam (BC), wurde 2016 im Auftrag eines englischen Partners entwickelt, für den deutschen Markt angepasst und wird inzwischen europaweit erfolgreich vertrieben.
Sie dient gegenwärtig in erster Linie zur Prävention und Deseskalation bei Konflikten in den Einsatzszenarien von Polizei, Ordnungsämtern, Verkehrsbetrieben und Sicherheitsdiensten, sowie zur deren Dokumentation, im Speziellen zur rechtskonformen Beweissicherung und deren Gerichtsverwertbarkeit.

Prävention und Deeskalation sind vor allem im Bereich Arbeitsschutz bzw., bei der Verhinderung von Arbeitsunfällen ein wichtiges Thema.
In Bahnbetrieben, zum Beispiel, fallen im Jahr 2020 knapp 27 \% und im Jahr 2021 über 21 \% aller Arbeitsunfälle in die Kategorie „Einwirkungen durch Gewalt, Angriff, Bedrohung“, davon über 60 \% durch betriebsfremde Personen, also z.B. Fahrgäste \cite[jeweils S.87 ff.]{Unfallgeschehen2020,Unfallgeschehen2021}.
Ob die abnehmende Tendenz mit dem Einsatz der Bodycams erklärt werden kann, ist noch nicht hinreichend evaluiert, lässt jedoch positive Auswirkungen annehmen.
Die NetCo Body -Cam wird gegenwärtig in zwei Versionen angeboten.
In der Record-Version wird die Aufnahme im Gerät gespeichert und nach dem Einsatz über eine Docking-Station in das zentrale System überspielt.
In der Connect-Version erfolgt die unmittelbare Übertragung über VPN an das zentrale System.
Beide Systemvarianten decken die Zielstellungen Prävention, Deeskalation und Dokumentation zuverlässig ab.

Durch verstärkte Kundenanfragen (evtl. Experteninterview einfügen) lassen sich zunehmend weitere Zielstellungen und damit Einsatzgebiete identifizieren.
So nimmt die Absicherung von Alleinarbeitern einen immer größeren Stellenwert ein.

Alleinarbeit (engl. Lonework) wird nach (DGVU §8) \cite[§8]{Vorschrift1_DGUV} unter bestimmten Umständen als gefährliche Arbeit eingestuft, z.B. Dienstleistungen an Personen, die sich gegen die Dienstleistung tätlich wehren, wie es in der mobilen Pflege, den Verkehrsbetrieben oder in der speziellen Jugendhilfe der Fall sein kann.
Unter 2.7.2 DGVU-Regel 100-001 „Regeln zur Prävention“ heißt es: „Wird eine gefährliche Arbeit von einer Person allein ausgeführt, so hat der Unternehmer über die allgemeinen Schutzmaßnahmen hinaus für geeignete technische oder organisatorische Personenschutzmaßnahmen zu sorgen (DGVU Vorschrift 1 §8(2) DGUV: Präv - DGUV Vorschrift 1 ).
Alleinarbeit liegt vor, wenn eine Person allein, außerhalb von Ruf- und Sichtweite zu anderen Personen, Arbeiten ausführt“ \cite[S.42]{Regel_100-001}.
Nach Geyer \& Magiera folgt diese Fürsorge einem gesetzlich geregeltem Schema aus den Bausteinen: Festlegung der Arbeitsbereiche und Tätigkeiten, Ermittlung der Gefährdungen und Festlegung konkreter Arbeitsschutzmaßnahmen, inklusive Umsetzung, sowie Wirkungskontrolle.\cite[vgl. S.42 ff.]{GeyerMagiera2022}.
Wird erhöhte oder kritische Gefährdung ermittelt, erfordert es Schutzmaßnahmen durch Überwachung in unterschiedlicher Ausprägung.

In dieser notwendigen Weiterentwicklung unseres Produktes soll die BC für spezielle Einsatzzwecke nun den sogenannten Loneworker-Modus als neues Feature bekommen.
Durch diesen Modus soll die Sicherheit des Trägers durch erweiterte Notrufmöglichkeiten im Notfall im Fokus stehen.
Weiterhin bietet die BC eine als entlastend empfundene mentale Unterstützung durch die geeignete Überwachung.
Nach Geyer \& Magiera ergeben sich neben den häufig ersichtlichen Gefährdungsfaktoren durch Arbeitsumgebungsbedingungen weitere Einflüsse auf die Beurteilung der Gefährdungen im Zusammenhang mit Alleinarbeit.
Unter anderem ist zu fragen, wie Alleinarbeit auf den Mitarbeitenden in Summe wirkt.
Dabei ist auch an psychische Belastungen durch Alleinarbeit zu denken, z. B.: die Vorstellung bzw. Angst, dass niemand zu Hilfe kommt, wenn etwas passiert, Stress wegen mangelnder Unterstützung bei außergewöhnlichen Ereignissen oder das Gefühl der Isolation (vgl. Geyer \& Magiera becks online ARP 2022 S.45).

Für dieses Wirkspektrum wird eine kontinuierliche Verbindung mit einer Überwachungszentrale bereitgestellt.
So können einerseits Sicherheit vermittelt werden und andererseits unerwartete Situationen erkannt und schnell reagiert werden.
Um die Verbindung und die damit verbundene Sicherheit festzustellen müssen beide Seiten in kurzer Zeit einen Verbindungsverlust erkennen und nach vorher definierten Handlungen reagieren, wie z.B. das Auslösen eines Alarms.
Damit einher müssen Prozeduren geschaffen werden, die einen sauberen und sicheren Übergang von aktivem und inaktivem Modus sicherstellen und diesen von einem unerwarteten Verbindungsverlust abgrenzen zu können.
Außerdem sollte bei einem plötzlichen Verbindungsabbruch die Kommunikation so schnell wie möglich und automatisch wieder hergestellt werden.
Da dies Mobilfunk oder WLAN nativ nicht zur Verfügung stellen.
Neben den vorhandenen BC-Funktionen (Aufnahme, Streaming), sollen durch den LW-Modus weitere Optionen geschaffen werden.
Darunter Alarmsignalisierung, Sprachkommunikation, Textnachrichten oder Geofencing.
Alle Ereignisse müssen sicher und nachvollziehbar protokolliert werden.

In dieser Arbeit soll dafür das Grundgerüst des LW-Modus gebildet werden.
Hierfür soll ein Kommunikationsprotokoll realisiert werden.
Außerdem dem soll die Basiskommunikation implementiert werden die mindestens aus Verbindungsaufbau, -abbau, -wiederaufbau und einer schnellen Abbrucherkennung besteht.
Für das Konzept soll Vollständigkeit und Richtigkeit nachgewiesen werden und die Implementierung soll getestet und damit eine Funktionierende Verbindung zwischen ARC und BC im LW-Modus nachgewiesen werden.
Alle weiteren Funktionen wie Alarmierung, Audio-, Videostreaming, Positionsüberwachung(Geofencing) sollen theoretisch diskutiert und Schnittstellen für diese definiert werden.

In der vorliegenden Bachelorarbeit wird das Thema "Lone Worker" in Verbindung mit der Nutzung von Bodycams und der Einbindung in ein Überwachungszentrum näher untersucht.
Ziel der Arbeit ist es, eine Lone-Worker-Verbindung zwischen einer Bodycam und einem Überwachungszentrum zu erarbeiten und die Wirksamkeit dieser Lösung als Sicherheitsmaßnahme für Lone Workers zu evaluieren.
Hierzu werden Herausforderungen und Risiken, mit denen Alleinarbeitende konfrontiert sind erörtert und aufgezeigt inwiefern der Einsatz von Bodycams mit einer aktiven Verbindung zu einem Überwachungszentrum dazu beitragen können diese Herausforderungen zu bewältigen und die Sicherheit der Lone Workers zu erhöhen.
Anschließend wird ein Konzept erstellt, welches ein Protokoll enthält, das die Kommunikation zwischen Bodycam und Überwachungszentrum realisieren soll.
Um die Verbindung und die damit verbundene Sicherheit festzustellen müssen beide Seiten in kurzer Zeit einen Verbindungsverlust identifizieren und nach vorher definierten Handlungen reagieren, wie z.B. das Auslösen eines Alarms.
Damit einher müssen Prozeduren geschaffen werden, die einen sauberen und sicheren Übergang von aktivem und inaktivem Modus sicherstellen und diesen von einem unerwarteten Verbindungsverlust abgrenzen zu können.
Außerdem sollte bei einem plötzlichen Verbindungsabbruch die Kommunikation so schnell wie möglich und automatisch wieder hergestellt werden.
Da dies Mobilfunk oder WLAN nativ nicht zur Verfügung stellen.
Neben vorhandenen Bodycam-Funktionen wie beispielweise Aufnahme und Streaming, sollen durch den Lone-Workers-Modus weitere Optionen geschaffen werden.
Darunter zählen Alarmsignalisierung, Sprachkommunikation, Textnachrichten oder Geofencing. Alle Ereignisse müssen sicher und nachvollziehbar protokolliert werden.

Die Bachelorarbeit gliedert sich in drei Hauptteile. Im ersten Teil werden die Anforderungen an das Gesamtsystem zwischen Bodycam mit seinem Lone Workers und dem Überwachungssystem erarbeitet. Außerdem werden die einzelnen Komponenten beschrieben und Use-Cases präsentiert. Im zweiten Teil wird das Konzept für die Verbindung des Lone Workers über die Bodycam mit einem Überwachungszentrum dargestellt. Hierbei fließen das Protokolldesign sowie eine Diskussion über die Anforderungserfüllung mit ein. Im dritten Teil werden die Ergebnisse der eigenen Untersuchung sowie die Resultate aus dem praktischen Implementierungsteil erklärt und gezeigt. Außerdem wird die Implementierung mit einzelnen Use-Cases verknüpft sowie die Methoden der agilen Entwicklung im Entwicklungsprozess diskutiert. Im vierten Teil werden Testfälle präsentiert und die nachfolgenden Funktionen diskutiert, die es zum vollständigen praktischen Einsatz der Lone Workers Verbindung braucht. Im letzten Teil werden schließlich die Ergebnisse zusammengefasst und Handlungsempfehlungen für die praktische Anwendung der Lone-Worker-Verbindung abgeleitet.


\section{Systembeschreibung der Komponenten} \label{bib:goals}



\section{Systembeschreibung der Use-Cases}
Describe all the novelties in your work.
\\
Next chapter ....
\\
\autoref{chap:grundlagen} discusses ...
\\
Our own contribution ... described in \autoref{chap:design}.
Results are evaluated and discussed in \autoref{chap:ergebnisse}.
\\
Finally, \autoref{chap:fazit} will summarize the thesis and give an outlook to possible future work.

\subfilebib % Makes bibliography available when compiling as subfile
\end{document}
\documentclass[thesis.tex]{subfiles}
\begin{document}

\chapter{Einleitung}
\label{chap:Einleitung}

% An image right here at the top can look really cool!
\begin{figure}[h]
    \centering
    % \includegraphics[height=0.5\textwidth]{/BC_activDisplay.png}
    \includegraphics[height=0.55\textwidth]{/BC_hand.png}
    \caption[NetCo Body-Cam]{NetCo Body-Cam \cite{netco}}
    \label{fig:BC}
\end{figure}

Das Unternehmen NetCo Professional Services GmbH (NetCo) mit Sitzen in Blankenburg und Magdeburg entwickelt und produziert individuelle Kamera- und Softwarelösungen.
Ein Hauptprodukt, die NetCo Body-Cam, wurde 2016 im Auftrag eines englischen Partners entwickelt, für den deutschen Markt angepasst und inzwischen europaweit erfolgreich vertrieben.
Sie dient gegenwärtig in erster Linie zur Prävention und Deeskalation bei Konflikten in den Einsatzszenarien von Polizeien, Ordnungsämtern, Verkehrsbetrieben und Sicherheitsdiensten sowie zu deren Dokumentation, im Speziellen zur rechtskonformen Beweissicherung und deren Gerichtsverwertbarkeit.

Prävention und Deeskalation sind vor allem im Bereich Arbeitsschutz bzw. bei der Verhinderung von Arbeitsunfällen ein wichtiges Thema.
In Bahnbetrieben, fielen im Jahr 2020 knapp 27 \% und im Jahr 2021 über 21 \% aller Arbeitsunfälle in die Kategorie \glqq Einwirkungen durch Gewalt, Angriff, Bedrohung\grqq{}, davon über 60 \% durch betriebsfremde Personen, wie z.~B. Fahrgäste \cite[vgl.~S.~87~ff.][]{Unfallgeschehen2020,Unfallgeschehen2021}.
Ob die abnehmende Tendenz mit dem Einsatz der Bodycams erklärt werden kann, ist noch nicht hinreichend evaluiert, lässt jedoch positive Auswirkungen vermuten.
Die NetCo Body-Cam wird gegenwärtig in zwei Versionen angeboten.
In der Record-Version wird die Aufnahme im Gerät gespeichert und nach dem Einsatz über eine Docking-Station in das zentrale System überspielt.
In der Connect-Version erfolgt die unmittelbare Übertragung über ein \glqq Virtual Private Network\grqq{} (VPN) an das zentrale System.
Beide Systemvarianten decken die Zielstellungen Prävention, Deeskalation und Dokumentation zuverlässig ab.
%% ABKÜRZUNG?

Durch verstärkte Kundenanfragen lassen sich zunehmend weitere Zielstellungen und damit Einsatzgebiete identifizieren.
So nimmt die Absicherung von Alleinarbeitern einen immer größeren Stellenwert ein.

Alleinarbeit (engl. Lonework) wird nach der Deutschen Gesetzlichen Unfallversicherung (DGUV) unter bestimmten Umständen als gefährliche Arbeit eingestuft, z.~B. Dienstleistungen an Personen, die sich gegen die Dienstleistung tätlich wehren, wie es in der mobilen Pflege, den Verkehrsbetrieben oder in der speziellen Jugendhilfe der Fall sein kann \cite[§~8][]{Vorschrift1_DGUV}.
Unter 2.7.2 DGUV-Regel 100-001 \glqq Regeln zur Prävention\grqq{} heißt es: \glqq Wird eine gefährliche Arbeit von einer Person allein ausgeführt, so hat der Unternehmer über die allgemeinen Schutzmaßnahmen hinaus für geeignete technische oder organisatorische Personenschutzmaßnahmen zu sorgen\grqq{} \cite[§~8~(2)][]{Vorschrift1_DGUV}.
\glqq Alleinarbeit liegt vor, wenn eine Person allein, außerhalb von Ruf- und Sichtweite zu anderen Personen, Arbeiten ausführt\grqq{} \cite[S.~42][]{Regel_100-001}.
Nach Geyer \& Magiera folgt diese Fürsorge einem gesetzlich geregeltem Schema aus den Bausteinen: Festlegung der Arbeitsbereiche und Tätigkeiten, Ermittlung der Gefährdungen und Festlegung konkreter Arbeitsschutzmaßnahmen, inklusive Umsetzung, sowie Wirkungskontrolle \cite[vgl.~S.~42~ff.][]{GeyerMagiera2022}.
Wird eine erhöhte oder kritische Gefährdung ermittelt, erfordert dies Schutzmaßnahmen durch Überwachung in unterschiedlicher Ausprägung.

In dieser notwendigen Weiterentwicklung des Produktes soll die Body-Cam für spezielle Einsatzzwecke nun den sogenannten Lone-Worker-Modus als neues Feature bekommen.
Durch diesen Modus soll im Notfall die Sicherheit des Trägers durch erweiterte Notrufmöglichkeiten im Fokus stehen.
Weiterhin bietet die Body-Cam eine als entlastend empfundene mentale Unterstützung durch die geeignete Überwachung.
So ist neben den häufig ersichtlichen Gefährdungsfaktoren (Arbeitsumgebungsbedingungen, Arbeitszeitpunkten) auch an psychische Belastungen durch Alleinarbeit zu denken.
Unter psychische Belastungen zählen z.~B. die Vorstellung das im Notfall niemand zu Hilfe kommt, Stress bei außergewöhnlichen Ereignissen oder das Gefühl der Isolation \cite[vgl.~S.~45][]{GeyerMagiera2022}.

Für dieses Wirkspektrum wird eine kontinuierliche Verbindung mit einer Überwachungszentrale bereitgestellt.
So können einerseits Sicherheit vermittelt und andererseits unerwartete Situationen erkannt und sofort reagiert werden.
Um die Verbindung und die damit verbundene durchgehende Überwachung sicherzustellen, müssen beide Seiten in kurzer Zeit einen Verbindungsverlust erkennen und nach vorher definierten Handlungen reagieren, wie z.~B. das Auslösen eines Alarms.
Damit einhergehend müssen Prozeduren geschaffen werden, die einen sauberen und sicheren Übergang von aktivem und inaktivem Modus sicherstellen und diesen von einem unerwarteten Verbindungsverlust abgrenzen können.
Außerdem sollte bei einem plötzlichen Verbindungsabbruch die Kommunikation schnellstmöglich und automatisch wieder hergestellt werden.
Dies stellen Mobilfunk oder \glqq Wireless Local Area Network\grqq{} (WLAN) nativ nicht zur Verfügung.
Neben den vorhandenen Funktionen der Body-Cam (Aufnahme, Streaming, Dokumentation), sollen durch den Lone-Worker-Modus weitere Optionen geschaffen werden.
Diese sollen sicherstellen, dass die Body-Cam alle nötigen Voraussetzungen erfüllt, um die Sicherheit einer allein arbeitenden Person zu unterstützen.
Alle Ereignisse muss die Body-Cam sicher und nachvollziehbar protokollieren.

Im Rahmen dieser Arbeit wird ein Kommunikationsprotokoll zur Realisierung der Lone-Worker-Verbindung zwischen Body-Cam und Überwachungszentrum entwickelt.
Im praktischen Teil wird dazu die Kommunikationsbasis in Form eines Prototyps implementiert, der eine einfache Verbindung zwischen Client und Server mithilfe des Protokolls realisiert.
Die bestehende Implementierung der Body-Cam soll im späteren Verlauf des Lone-Worker-Projekts durch das Kommunikationsprotokoll erweitert werden können.
Der Prototyp bildet hierfür die Grundlage.
Die Integration der höheren Funktionen wie beispielsweise die Alarmsignalisierung, Geräteüberwachung, Sprachkommunikation, Video- und Audiostreaming, Textnachrichten oder Geofencing werden in der Designphase beschrieben und theoretisch diskutiert.
Sie sind jedoch nicht Teil der Implementierung.
Daraus ergeben sich folgende Punkte für die Entwicklung eines Protokolls, die innerhalb der Arbeit behandelt werden:
\begin{itemize}
    \item Verbindungsaufbau und -abbau,
    \item Abbrucherkennung (Heartbeats),
    \item Wiederherstellung der Verbindung nach Verlust,
    \item Senden und Empfangen von einheitlichen Nachrichten,
    \item Antworten auf eingehende Anfragen.
\end{itemize}

Die vorliegende Bachelorarbeit gliedert sich in drei Hauptteile.
Im ersten Teil werden die Grundlagen für die weitere Arbeit gelegt.
Es wird definiert was Alleinarbeit ist und wie sich diese in verschiedene Gefährdungsstufen einteilen lässt.
Außerdem welche aktuellen Alleinarbeitslösungen es gibt und wie sich die Body-Cam mit dem Lone-Worker-Modus dort einordnen lässt.
Anschließend wird der aktuelle Zustand der Body-Cam aufgezeigt und wie dieser durch die neue Lone-Worker-Funktion erweitert wird.
Das sich daraus ergebene Gesamtsystem wird beschrieben und dessen Anforderungen herausgearbeitet.
Diese ergeben sich aus den DGUV-Normen und zusätzlichen Unternehmensanforderungen.
Im zweiten Teil wird das Designkonzept behandelt.
Dafür werden aus den Anforderungen die verschiedenen Funktionen erarbeitet, die in den Lone-Worker-Modus der Body-Cam einfließen.
Das Protokoll und dessen Aufbau wird erläutert, um eine Kommunikation zwischen der Body-Cam und dem Überwachungszentrum sicherzustellen.
Außerdem werden alle Schnittstellen beschrieben, die sich zwischen der bestehenden Body-Cam Implementierung und dem neuen Lone-Worker-Modus ergeben.
% Zusätzlich wird durch Statusmaschinen die Implementierung der Funktionen in ein einfaches Schema gebracht und die Realisierung damit vorbereitet.
Der letzte Teil behandelt die Realisierung des Projekts.
Zuerst werden die verwendeten Technologien und technischen Grundlagen erklärt.
Danach folgt die Beschreibung des Entwicklungsprozesses und die Vorstellung der Implementierung.
Anschließend wird der zuvor entwickelte Prototyp durch ein Testprotokoll auf seine Funktionalität geprüft und die Ergebnisse werden ausgewertet.


% \autoref{chap:grundlagen} discusses ...
% \\
% Our own contribution ... described in \autoref{chap:design}.
% Results are evaluated and discussed in \autoref{chap:ergebnisse}.
% \\
% Finally, \autoref{chap:fazit} will summarize the thesis and give an outlook to possible future work.

\subfilebib % Makes bibliography available when compiling as subfile
\end{document}
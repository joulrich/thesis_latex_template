\documentclass[thesis.tex]{subfiles}

\begin{document}

\chapter{Fazit}\label{chap:fazit}

% \section{Ausblick}

% In dieser Arbeit wurde gezeigt welche Schritte notwendig sind, um die aktuelle Lösung der Body-Cam um einen Lone-Worker-Modus zu erweitern.
% Dazu wurde beschrieben, wie Lösungen für Alleinarbeit funktionieren und wie sie beispielsweise in Form eines PNAs eingesetzt werden.
% Alle von der Body-Cam benötigten Funktionen, wurden erläutert und sind in den Designprozess eines neuen Lone-Worker Feature mit eingeflossen.
% Im praktischen Teil wurde ein erster Prototyp als Grundlage für eine Kommunikation zwischen Body-Cam und Überwachungszentrum implementiert.
% Es gibt jedoch viele Sachverhalte die weiter ausgearbeitet werden müssen, um die Body-Cam erfolgreich als Lösung für Alleinarbeit einsetzten zu können.

% Auf der Grundlage des Prototyps müssen alle in der Arbeit beschriebenen Funktionen vollständig entwickelt werden.
% Darunter zählen beispielsweise die Alarmierung, verschiedene Formen von Feedback oder die Zweiwege Sprachkommunikation.
% Diese Aspekte müssen in die vorhandene Struktur der Body-Cam eingebunden und über die beschriebenen Primitiven ansprechbar gemacht werden.
% Auf Seite des Überwachungszentrums muss die Serverbasis ebenfalls erweitert werden.
% Dazu ist es nötig, die Struktur der Anfragen auszubauen, um damit eine Möglichkeit zu schaffen, alle Anforderungen an ein PNA zu testen.
% Hierzu zählen zum Beispiel das Anfragen und Anzeigen von Video- und Audiostreams oder die Standortüberwachung.

% In Verbindung mit dem Einsatz in anderen Unternehmen oder im öffentlichen Dienst muss für diesen Zweck die Infrastruktur geschaffen werden.
% Dafür können bestehende Überwachungszentrum auf die Body-Cam angepasst werden oder neue gebaut werden.
% In erster Linie bieten sich Unternehmen an die bereits Erfahrungen im Bereich der Alleinarbeit haben.
% Mithilfe dieser kann das neue Produkt in mehreren Probeläufen getestet und verbessert werden.
% Als erste Einsatzgebiete eignen sich Arbeiten der geringen Gefährdung, wie zum Beispiel der Einsatz bei LKW-Fahrten im Werksverkehr.
% Bei diesen ist das Gefahrenpotential geringer und es können alle Funktionen und Szenarien geprüft werden.
% Endgültig soll die Body-Cam, mit dem neuen Lone-Worker-Modus, aber in dem Bereich der \glqq gefährlichen Arbeiten\grqq{} verwendet werden, da sich hier ihre Stärken am besten einsetzten lassen.

% Ein weiterer wichtiger Punkt ist der rechtliche Rahmen.
% Durch den Einsatz als Überwachungsgerät mit Video und Audio ergibt sich eine große Verantwortung.
% So müssen in privaten Unternehmen, aber vor allem in öffentlichen Bereichen, klare Regeln festgelegt und verschiedene Gesetze beachtet werden.
% Beispielsweise steht NetCo in engem Kontakt mit Ordnungsämtern und den Ländern, um den Einsatz der aktuellen Body-Cam in allen Bundesländern zu ermöglichen.
% Dies birgt jedoch viele Hürden, da momentan jedes Bundesland eigene Regelungen zum Einsatz von Body-Cams hat.

% \section{Zusammenfassung}

Das Ziel der vorliegenden Arbeit war es, die bereits erfolgreich eingesetzten Body-Cam von NetCo, um den Lone-Worker-Modus zu erweitern.
Dafür wurde ein Konzept entwickelt, um die bestehenden Funktionalitäten für Prävention, Deeskalation und Protokollierung auszubauen.
Bereits in der Ursprungsversion erfüllt die Body-Cam die wesentlichen Funktionen einer PNA.
% Insbesondere für den Einsatz bei gefährlichen Arbeiten.
Die Hardware muss um einen Alarmknopf und einen Lagesensor mit den entsprechenden Funktionalitäten ergänzt werden.
Idealerweise erweitert die bereits vorhandene Kamera das Wirkspektrum herkömmlicher PNAs sinnvoll und bietet eine höhere Sicherheit im Einsatzbereich gefährlicher Alleinarbeit.
Wesentliche Grundlage dieser Funktionserweiterung sind die in dieser Arbeit beschriebenen Softwarefunktionen sowie der implementierte Prototyp.
Letzteres bildet die Basis für eine ununterbrochene Client-Server-Verbindung zwischen Body-Cam und Überwachungszentrum.
Der Lone-Worker-Modus wurde beschrieben, designt und als Vorstufe in Form des Prototyps entwickelt.
Schwerpunkt bildete dabei die Client-Seite der Verbindung.

Es wurde beschrieben, wie Lösungen für Alleinarbeit funktionieren und wie sie beispielsweise in Form eines PNAs eingesetzt werden.
Alle von der Body-Cam benötigten Funktionen, wurden erläutert und sind in den Designprozess eines neuen Lone-Worker-Features eingeflossen.

Im praktischen Teil wurde ein erster Prototyp als Grundlage für eine zuverlässige Kommunikation zwischen Body-Cam und Überwachungszentrum erfolgreich implementiert.
Um den Lone-Worker-Modus mit seinen komplexen Funktionalitäten vollständig abzubilden, müssen alle höheren Funktionen entwickelt und eingeführt werden.
Diese wurden in der Arbeit theoretisch beschrieben, sind an dieser Stelle jedoch nicht praktischer Teil der Entwicklung.
Darunter zählen beispielsweise die Alarmierung, verschiedene Formen von Feedback oder die Zweiwege-Sprachkommunikation.
Diese Aspekte müssen zukünftig in die vorhandene Struktur der Body-Cam eingebunden und über die beschriebenen Primitiven ansprechbar gemacht werden.

Auf Seite des Überwachungszentrums muss die Serverbasis ebenfalls erweitert werden.
Dazu ist es nötig, die Struktur der Anfragen auszubauen, um damit eine Möglichkeit zu schaffen, alle Anforderungen an diese PNA zu testen.
Hierzu zählen zum Beispiel das Anfragen und Anzeigen von Video- und Audiostreams oder die Standortüberwachung in Verbindung mit Geofencing.
Mit dem Einsatz in anderen Unternehmen oder im öffentlichen Dienst muss für diesen Zweck die Infrastruktur geschaffen werden.
Dafür können bestehende Überwachungszentrum auf die Body-Cam angepasst werden oder neue gebaut werden.

In erster Linie bieten sich Unternehmen an, die bereits Erfahrungen im Bereich der Alleinarbeit haben.
Mithilfe dieser kann das neue Produkt in mehreren Probeläufen getestet und verbessert werden.
Als erste Einsatzgebiete eignen sich Arbeiten der geringen Gefährdung, wie zum Beispiel der Einsatz bei LKW-Fahrten im Werksverkehr.
Bei diesen ist das Gefahrenpotential geringer und es können alle Funktionen und Szenarien geprüft werden.
Endgültig soll die Body-Cam, mit dem neuen Lone-Worker-Modus, aber in dem Bereich der \glqq gefährlichen Arbeiten\grqq{} verwendet werden, da hier ihre Stärken am besten wirksam werden.

Ein nicht zu vernachlässigender Punkt ist der rechtliche Rahmen.
Durch den Einsatz als Überwachungsgerät mit Video und Audio ergibt sich eine große Verantwortung.
So müssen in privaten Unternehmen, aber vor allem in öffentlichen Bereichen, klare Regeln festgelegt und verschiedene Gesetze beachtet werden.
Beispielsweise steht NetCo in engem Kontakt mit Ordnungsämtern und den Ländern, um den Einsatz der aktuellen Body-Cam in allen Bundesländern zu ermöglichen.
Dies birgt jedoch einige Hürden, da momentan jedes Bundesland eigene Regelungen zum Einsatz von Body-Cams hat.

Die steigende Nachfrage nach komplexen Lösungen und vor allem der Sicherheitsgewinn für Beschäftigte rechtfertigt aus Sicht des Autors den Aufwand der Weiterentwicklung und aufwändigen rechtlichen Prüfung.


\subfilebib % Makes bibliography available when compiling as subfile
\end{document}